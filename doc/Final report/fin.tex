\documentclass[titlepage]{article}
\title{The Undecided - Final Report - 7CCSMGRP}
\author{Nathalie Caliacmani, Amar Menezes, \\ Belema Norman-William, Paul Orlean-Taub, Tony Sellen}
\date{March 2015}

\begin{document}
\maketitle

\begin{abstract}
Since it's development, the car has had a huge economic and social influence. It has created industries and caused others to crumble. The popularity of the auto-mobile has had an impact on city planning due to its effects on employment, trade, manufacture and even basic social interaction.

As the popularity of these vehicles in both a private and public context has increased, transport planners and civil engineers have had to adjust our road networks accordingly often trough the development of mathematical models for traffic flow (of varying accuracy).

This project is our attempt at producing a software that shows the effects of changes to pre-existing networks (i.e. increase/decrease in demand, adding lanes, changing existing lanes into bus lanes) on the overall flow. The primary output will be data that can be fed into an appropriate statistical package for professional analysis.
\end{abstract}

\tableofcontents
\newpage

\section{Introduction}
	In 1769, Cugnot developed the first steam powered auto-mobile \cite{eckermann2001world} (self propelled vehicle not operating on tracks). As these vehicles spread to the UK, the government passed the locomotive acts providing a framework for the automotive industry here to develop and regulating the use of these machines; The act of 1861 imposed speed limits, determined tolls, set fines, and stated design requirements; the act of 1865 tightened these regulations and required that a man waving a red flag would be required to walk ahead of all of these vehicles; and the 1878 act assigning the powers.
	
	In 1886\footnote{The same year Coca Cola was invented} Karl Benz filed his patent for the petrol powered ``moterwagen'' \cite{benzpatent} sparking the production of uniform vehicles that could travel long distances. The uptake of these machines leading to further regulations and the requirement for licensing. From this time onwards there was a steady increase in car ownership, primarily amongst the wealthy and private businesses.

	In 1908, the Ford Motor Company developed the Ford Model T, and with it, the automatic assembly line allowing the production of these cars in 93 minutes. The increase in production speed allowed costs to be optimised and made the car affordable to the mass market.
	
	In the years that followed, cars became a primary mode of transportation. In response to increased usage, the British government passed laws that were less restrictive on car use, and optimised road networks to cope with an increase in vehicles that were larger, heavier, and faster than those for which they were originally developed.
	
	Nowadays cars are widespread, and finely engrained in our culture\cite{miller2001car}\cite{berdahl2000go}. However new developments lead to large changes in ownership rates and road demand, and many economies rely on the optimisation of traffic flow, with traffic engineers often needing to change factors in order to artificially increase and decrease congestion \cite{blunden1983congestion}. These engineers use mathematical models to simulate flow through a network and often need to trial the effects of various changes.
	
	This project is our attempt at producing a piece of software to model a road network and create artificial (and hopefully realistic) data that will allow a civil engineer to perform the appropriate analysis and determine the best way to optimise our pre-set systems 

\section{Review}

\section{Requirements and Design}

\section{Implementation}

\section{Teamwork}

\section{Evaluation}

\section{Peer Assessment}
	\begin{center}
	\begin{tabular}{| l | c |}
		\hline
		\textbf{Name}					&	\textbf{Points} \\ \hline
		Nathalie Caliacmani				&	20				\\ \hline
		Amar Menezes					&	20				\\ \hline
		Belema (Emily) Norman-William	&	20				\\ \hline
		Paul Orlean-Taub				&	20				\\ \hline
		Anthony Sellen					&	20				\\ \hline
		\textbf{Sum}					&	\textbf{100}	\\
		\hline
	\end{tabular}
	\end{center}
		

\bibliographystyle{plain}
\bibliography{bibfile}
\end{document}