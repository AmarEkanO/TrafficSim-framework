In 1769, Cugnot developed the first steam powered auto-mobile \cite{eckermann2001world} (self propelled vehicle not operating on tracks). As these vehicles spread to the UK, the government passed the locomotive acts providing a framework for the automotive industry here to develop and regulating the use of these machines; The act of 1861 imposed speed limits, determined tolls, set fines, and stated design requirements; the act of 1865 tightened these regulations and required that a man waving a red flag would be required to walk ahead of all of these vehicles; and the 1878 act assigning the powers.
	
	In 1886\footnote{The same year Coca Cola was invented} Karl Benz filed his patent for the petrol powered ``moterwagen'' \cite{benzpatent} sparking the production of uniform vehicles that could travel long distances. The uptake of these machines leading to further regulations and the requirement for licensing. From this time onwards there was a steady increase in car ownership, primarily amongst the wealthy and private businesses.

	In 1908, the Ford Motor Company developed the Ford Model T, and with it, the automatic assembly line allowing the production of these cars in 93 minutes. The increase in production speed allowed costs to be optimised and made the car affordable to the mass market.
	
	In the years that followed, cars became a primary mode of transportation. In response to increased usage, the British government passed laws that were less restrictive on car use, and optimised road networks to cope with an increase in vehicles that were larger, heavier, and faster than those for which they were originally developed.
	
	Nowadays cars are widespread, and finely engrained in our culture\cite{miller2001car}\cite{berdahl2000go}. The representation of traffic systems and flow is an essential part of planning transportation systems or setting up a road network. As a critical tool for government transport authority and consultants, urban traffic models have been given a great deal of consideration by analysts and different experts. And the reason for this is because traffic-jam or bottleneck adds to the complexity of traffic models, it is also important for non-urban speculation and transportation planning systems. The flow of traffic in a system might be dealt with as dynamic, without considering the independent components, or vehicles may be modelled. 
	
	Our project "Traffic  Regulation And Testing Tool" (TRATT) is our attempt at producing a piece of software to model a road network and create artificial (and hopefully realistic) data that will allow a civil engineer to perform the appropriate analysis and determine the best way to optimise infrastructure and policy changes before deploying them in real life.